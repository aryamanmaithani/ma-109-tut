\documentclass[12pt]{article}
\usepackage{amsmath, amssymb, amsfonts, amsthm, mathtools,mathrsfs}
\usepackage{thmtools}
\usepackage[utf8]{inputenc}
\usepackage[inline]{enumitem}
\usepackage[colorlinks=true]{hyperref}
\usepackage{tikz}
\usetikzlibrary{decorations.markings}
\usetikzlibrary{arrows.meta}
\usepackage{witharrows}
\usepackage{datetime2}

\setlength\parindent{0pt}
\usepackage{parskip}

\usepackage[framemethod=tikz]{mdframed}
\mdfdefinestyle{theoremstyle}{%
	% linecolor=gray,linewidth=1pt,%
	% frametitlerule=true,%
	frametitlebackgroundcolor=white,
	% backgroundcolor=  gray!20,	
	bottomline=false, topline=false, rightline=false, leftline=true,
	innerlinewidth=0.7pt, outerlinewidth=0.7pt, middlelinewidth=2pt, middlelinecolor=white, %
	innerleftmargin=6pt,
	% innertopmargin=-1pt,
	skipabove=10pt,
	% fontcolor=blue,
	% innerbottommargin=-0.5pt,
}
\mdtheorem[style=theoremstyle]{defn}[thm]{Definition}
\mdtheorem[style=theoremstyle]{thm}{Theorem}

\newcommand*{\doublerule}{\hrule width \hsize height 1pt \kern 0.5mm \hrule width \hsize height 2pt}
\newcommand{\doublerulefill}{\leavevmode\leaders\vbox{\hrule width .1pt\kern1pt\hrule}\hfill\kern0pt}

\usepackage{xpatch}
\newcommand{\thmautorefname}{Theorem}
\makeatletter
\xpatchcmd{\thm}{\refstepcounter}{\NR@gettitle{#1}\refstepcounter}{}{}
\makeatother

\newcommand{\Res}{\operatorname{Res}}

\theoremstyle{definition}
% \newtheorem{thm}{Theorem}
% \numberwithin{thm}{section}
% \newtheorem{lem}[thm]{Lemma}
% \newtheorem{defn}[thm]{Definition}
% \newtheorem{prop}[thm]{Proposition}
% \newtheorem{cor}[thm]{Corollary}
% \newtheorem{ex}{Example}


\let\emptyset\varnothing

\pagestyle{plain}

\usepackage{titlesec}
\titleformat{\section}[block]{\sffamily\Large\filcenter\bfseries}{\S\thesection.}{0.25cm}{\Large}
\titleformat{\subsection}[block]{\large\bfseries\sffamily}{\S\S\thesubsection.}{0.2cm}{\large}

\usepackage[a4paper]{geometry}
\usepackage{lipsum}

\usepackage{cleveref}
\crefname{thm}{Theorem}{Theorems}
\crefname{lem}{Lemma}{Lemmas}
\crefname{defn}{Definition}{Definitions}
\crefname{prop}{Proposition}{Propositions}
\crefname{cor}{Corollary}{Corollaries}
\crefname{equation}{}{}

\usepackage{mdframed}
\newenvironment{blockquote}
{\begin{mdframed}[skipabove=0pt, skipbelow=0pt, innertopmargin=4pt, innerbottommargin=4pt, bottomline=false,topline=false,rightline=false, linewidth=2pt]}
{\end{mdframed}}
\newenvironment{soln}{\begin{proof}[Solution]}{\end{proof}}

\usepackage{fancyhdr}
\setlength{\headheight}{15.2pt}
\pagestyle{fancy}
\fancyhf{}
\fancyhead[L]{\sffamily{\S\textbf{\nouppercase{\leftmark}}}}
\fancyhead[R]{\sffamily{\thepage}}
\definecolor{myupdatecolor}{RGB}{0, 0, 255}

\newcommand{\remark}[1]{{\color{gray}[#1]}}
\newcommand{\md}[1]{\left\lvert #1 \right\lvert}
\newcommand{\fl}[1]{\left\lfloor #1 \right\rfloor}

% \usepackage{xcolor}
% \definecolor{mybgcolor}{RGB}{50, 50, 50} %46, 51, 63
% \usepackage{pagecolor}
% \pagecolor{mybgcolor}
% \color{white}
% \mdfsetup{backgroundcolor=mybgcolor, fontcolor=white}
% \definecolor{myupdatecolor}{RGB}{0, 255, 0}

\renewcommand{\familydefault}{\sfdefault}

\title{MA 109: Calculus I\\\large{Assignment Solutions}}
\author{Aryaman Maithani\\\url{https://aryamanmaithani.github.io/tuts/ma-109}}
\date{Autumn Semester 2020-21\\~\\Last update: \DTMnow}

\begin{document}
\maketitle
\tableofcontents
\newpage
\section{Assignment 1}
Take the last digit of your roll number. Call it $w.$ Take the second last digit of your roll number. Call it $z.$ Let $a = w + 10$ and $b = z + 10.$ Evaluate the limit
\begin{equation*} 
	\lim_{n\to \infty}\dfrac{an + 1}{bn + 2}.
\end{equation*}	
Justify your answer using the $\epsilon-N$ definition of the limit.

\begin{soln}
	Calculate $a$ and $b$ first, as per \textbf{your} roll number (and not the roll number of the person you're copying from).

	\textbf{Claim.} We claim that $\displaystyle\lim_{n\to \infty}\dfrac{an + 1}{bn + 2} = \dfrac{a}{b}.$ 

	\remark{You don't have to justify how you came up with the limit.}

	We need to show that given \emph{any} $\epsilon > 0,$ there exists $N \in \mathbb{N}$ such that
	\begin{equation*} 
		\md{\dfrac{an + 1}{bn + 2} - \dfrac{a}{b}} < \epsilon\quad\text{for all } n > N.
	\end{equation*}
	\remark{Note the order of $\epsilon, N,$ and $n.$ Changing this will change the meaning and won't be correct. Also note the ``for all'' in the two places.}

	To this end, given any $\epsilon > 0$ define $N \vcentcolon= \fl{\dfrac{2a - b}{b^2\epsilon}} + 1.$
	
	Now, note that if $n > N,$ then
	\begin{align*} 
		\hspace{2cm}\md{\dfrac{an + 1}{bn + 2} - \dfrac{a}{b}} &= \dfrac{2a - b}{b(bn + 2)} & (\because b - 2a < 0)\\
		&< \dfrac{2a - b}{b^2n}\\
		&< \dfrac{2a - b}{b^2N}\\
		&< \epsilon,
	\end{align*}
	as desired.
\end{soln}

An alternate:

\begin{soln}
	Calculate $a$ and $b$ first, as per \textbf{your} roll number (and not the roll number of the person you're copying from).

	\textbf{Claim.} We claim that $\displaystyle\lim_{n\to \infty}\dfrac{an + 1}{bn + 2} = \dfrac{a}{b}.$ 

	\remark{You don't have to justify how you came up with the limit.}

	We need to show that given \emph{any} $\epsilon > 0,$ there exists $N \in \mathbb{N}$ such that
	\begin{equation*} 
		\md{\dfrac{an + 1}{bn + 2} - \dfrac{a}{b}} < \epsilon\quad\text{for all } n > N.
	\end{equation*}
	\remark{Note the order of $\epsilon, N,$ and $n.$ Changing this will change the meaning and won't be correct. Also note the ``for all'' in the two places.}

	To this end, given any $\epsilon > 0$ and $n \in \mathbb{N},$ we note that
	
	\begin{equation*} 
	\begin{WithArrows}[displaystyle]
		\md{\dfrac{an + 1}{bn + 2} - \dfrac{a}{b}} < \epsilon &\iff \md{\dfrac{b - 2a}{b(bn + 2)}} < \epsilon \Arrow{$b - 2a < 0$ for your roll number}\\
		&\iff \dfrac{2a - b}{b(bn + 2)} < \epsilon \Arrow{since $n \in \mathbb{N}$ and hence, \\
		$b(bn + 2) > b(bn)$}\\
		&{\color{red}\impliedby} \dfrac{2a - b}{b^2n} < \epsilon
	\end{WithArrows}
	\end{equation*}

	Now, given any $\epsilon > 0,$ we choose $N \in \mathbb{N}$ such that
	\begin{equation*} 
		N > \dfrac{2a - b}{b^2\epsilon}.
	\end{equation*}
	\remark{You can also be explicit and choose $N = \fl{\dfrac{2a - b}{b^2\epsilon}} + 1$ but make sure your quantity is then actually a \underline{positive integer}.}

	Thus, for $n > N,$ we have
	\begin{equation*} 
		\dfrac{2a - b}{b^2n} < \epsilon
	\end{equation*}
	and thus, by our earlier observation, we see
	\begin{equation*} 
		\md{\dfrac{an + 1}{bn + 2} - \dfrac{a}{b}} < \epsilon
	\end{equation*}
	for all $n > N,$ as desired.
\end{soln}
\newpage
\subsection{Common mistakes}
\begin{enumerate}
	\item Not exactly a mistake but many of you spent a page or two in first ``finding'' the limit using Sandwich theorem and/or using that $1/n \to 0.$ This is unnecessary. 
	\item If you have an inequality like $A - C < B - C,$ you cannot (in general) conclude that $\md{A - C} < \md{B - C}.$ In fact, in this case, $A - C$ was usually negative and thus, you need to justify more.
	\item The direction of implication signs was in the opposite direction for many. (Note the red implication sign in the second solution. That's how it should be.) If you write something like $\md{a_n - l} < \epsilon \implies n > 1/\epsilon,$ then simply choosing $N > 1/\epsilon$ does not solve the problem because you haven't said that $n > 1/\epsilon \implies \md{a_n - l} < \epsilon.$
	\item \textbf{Don't} write something like $N = \dfrac{2a - b}{b^2\epsilon}.$ You need $N$ to be a positive \underline{integer}.
	\item Make sure you mention $\epsilon {\color{red}\;> 0}$.
	\item Note the definition says \underline{for all} $\epsilon > 0.$ You cannot choose $\epsilon$ on your own if you want to \emph{prove} a limit. Something like ``Set $\epsilon = \ldots$'' is incorrect.
	\item In the same vein as above, after you've fixed an $N \in \mathbb{N},$ the $\md{a_n - l} < \epsilon$ condition should hold for \underline{all} $n > N$ and not \emph{some}.
	\item Speaking of fixing an $N,$ do fix an $N$!\footnote{Exclamation, not factorial.} Many of you have not done that.
	\item Also, don't mess up the order of ``for every $\epsilon > 0,$ there exists $N \in \mathbb{N}$'' by writing ``there exists $N \in \mathbb{N}$ such that for every $\epsilon > 0$.'' The latter implies that a \emph{single} $N \in \mathbb{N}$ works for \emph{all} $\epsilon > 0.$ That is \emph{not} the case here.
	\item Instead of writing ``Assume that the limit is ...,'' you should write ``We claim that the limit is ...'' because you immediately follow it up with a proof. (The ``assuming'' thing is more appropriate when you want to disprove something by using contradiction.)
\end{enumerate}
%
%
%
\newpage
\section{Assignment 2}
Do there exist functions with the following properties? Justify.
\begin{enumerate}[leftmargin=*]
	\item $f:[0, 1] \to \mathbb{R}:$ convex and differentiable, $f'\left(\frac{1}{4}\right) = 2,$ $f'\left(\frac{3}{4}\right) = -1.$
	\item $f:[0, 1] \to \mathbb{R}:$ concave and discontinuous at $\frac{1}{2}.$
	\item $f:[0, 1] \to \mathbb{R}:$ convex and differentiable and such that $f'$ is not differentiable at $\frac{1}{2}.$
	\item $f:\mathbb{R} \to \mathbb{R}:$ strictly convex and strictly decreasing.
\end{enumerate}
\begin{soln}
	Note that if your answer is ``Yes,'' then you must give an example and justify why it has the given properties. Otherwise, give a proof that no such function can exist.
	\begin{enumerate}[leftmargin=*]
		\item {\color{red}No}.

		Suppose not. Let $f$ be a function with the given properties. 

		Since $f$ is convex and differentiable, $f'$ must be increasing.

		Since $\frac{1}{4} \le \frac{3}{4},$ we must have $f'\left(\frac{1}{4}\right) \le f'\left(\frac{3}{4}\right).$ Since 
		\begin{equation*} 
			f'\left(\frac{3}{4}\right) = -1 < 2 = f'\left(\frac{1}{4}\right),
		\end{equation*} we get a contradiction.
		%
		%
		\item {\color{red}No}.

		If a function $f:[a, b] \to \mathbb{R}$ is convex/concave, then it is continuous on $(a, b).$ Here, we have $a = 0$ and $b = 1.$ Since $\frac{1}{2} \in (0, 1),$ we see that no such function is possible.

		\remark{Note that this isn't given very precisely in slides. It is not true that $f$ will be continuous on $[0, 1].$ It may very well be discontinuous at the end-points. However, this time, no marks were deducted even if you wrote that ``Convex/cave functions are continuous.''}
		%
		%
		\item {\color{blue}Yes}.

		Consider $f:[0, 1] \to \mathbb{R}$ defined as
		\begin{equation*} 
			f(x) \vcentcolon= \begin{cases}
				0 & 2x \le 1,\\
				\left(x - \dfrac{1}{2}\right)^2 & 2x > 1.
			\end{cases}
		\end{equation*}
		Differentiability of $f$ at any point in $[0, 1]\setminus\left\{\frac{1}{2}\right\}$ is clear. At $1/2,$ we compute the LHD and RHD as follows:
		\begin{align*} 
			\lim_{h\to 0^+}\dfrac{f\left(\frac{1}{2} + h\right) - f\left(\frac{1}{2}\right)}{h} &= \lim_{h\to 0^+}\dfrac{f\left(\frac{1}{2} + h\right) - 0}{h}\\
			&= \lim_{h\to 0^+}\dfrac{\left(\dfrac{1}{2} + h - \dfrac{1}{2}\right)^2 - 0}{h}\\
			&= \lim_{h\to 0^+}\dfrac{h^2}{h} = 0,
		\end{align*}
		\begin{align*} 
			\lim_{h\to 0^-}\dfrac{f\left(\frac{1}{2} + h\right) - f\left(\frac{1}{2}\right)}{h} &= \lim_{h\to 0^-}\dfrac{f\left(\frac{1}{2} + h\right) - 0}{h}\\
			&= \lim_{h\to 0^-}\dfrac{0 - 0}{h}\\
			&= 0.
		\end{align*}
		Thus, we see that $f'\left(\frac{1}{2}\right) = 0.$ For the other points, we use the usual formulae to get
		\begin{equation*} 
			f'(x) \vcentcolon= \begin{cases}
				0 & 2x \le 1,\\
				2\left(x - \dfrac{1}{2}\right) & 2x > 1.
			\end{cases}
		\end{equation*}
		To see that $f'$ is not differentiable at $\frac{1}{2},$ we may compute the LHD and RHD again. One sees that the LHD is $0$ whereas the RHD is $2.$

		Moreover, note that $f'$ above is increasing, this tells us that $f$ is convex.
		%
		%
		\item {\color{blue}Yes}.

		Consider $f:\mathbb{R}\to\mathbb{R}$ defined by $f(x) \vcentcolon= \exp(-x).$ Note that $f$ is twice differentiable and that the following holds for all $x \in \mathbb{R}:$
		\begin{align*} 
			f'(x) &= -\exp(-x) < 0,\\
			f''(x) &= \exp(-x) > 0.
		\end{align*}
		The first inequality tells us that $f$ is strictly decreasing while the second tells us that $f$ is strictly convex. \qedhere
	\end{enumerate}
\end{soln}
\newpage
\subsection{Common mistakes}

\textbf{General}
\begin{enumerate}[label = \roman*.]
	\item If your answer is ``Yes,'' you cannot conclude by simply saying that ``The definition of {\color{red}foo} does not prevent {\color{blue}bar} and hence, it is possible.'' You must explicitly construct a example.
	\item You must justify your example in the above case. You cannot just state the function without explanation.
	\item You cannot simply draw a graph and justify your answer. Especially if your answer is ``No.''
\end{enumerate}

\textbf{Specific}
\begin{enumerate}[leftmargin=*, label = Q\arabic*.]
	\item You can't conclude by invoking $f''.$ You don't know if it exists.
	\item \phantom{hi}
	\begin{enumerate}[leftmargin=*, label = \roman*.]
		\item Some of you tried drawing graphs like the following to disprove it.

		\begin{center}
		
		\begin{tikzpicture}[scale=5,
		  mydot/.style={
		    circle,
		    fill=white,
		    draw,
		    outer sep=0pt,
		    inner sep=1.5pt
		  },
		  yourdot/.style={
		    circle,
		    fill=black,
		    draw,
		    outer sep=0pt,
		    inner sep=1.5pt
		  }
		]

		\def\shift{1.1};
		\draw[variable=\t,domain=0:1,samples=500] plot ({\t},{-2*(\t - 0.5)*(\t - 0.5)});
		 \draw (0.5, 0) node[mydot] {};
		 \draw (0.5, -0.1) node[yourdot] {};

		 \draw[variable=\t,domain=0:1,samples=500] plot ({\shift + \t},{-2*(\t - 0.5)*(\t - 0.5)});
		 \draw (\shift + 0.5, 0) node[mydot] {};
		 \draw (\shift + 0.5, 0.1) node[yourdot] {};
		\end{tikzpicture}
		\end{center}

		Firstly, graphs are not proofs. Even ignoring that, all you have really shown is that this particular discontinuous function isn't concave. That doesn't prove anything.
		%
		%
		\item Some of you have tried arguing using:
		\begin{center}
			Concavity $\implies$ Double derivative exists $\implies$ Function is continuous
		\end{center}

		This is not correct since concave functions don't even have to be differentiable. Consider $x \mapsto -|x|$ defined on $[-1, 1].$
	\end{enumerate}
	%
	%
	\item \phantom{hi}
	\begin{enumerate}[leftmargin=*, label = \roman*.]
		\item This has got to be my favourite mistake I've (possibly ever) seen being made. Some of you have defined the function ``piecewise'' as:
		\begin{equation*} 
			f(x) = \begin{cases}
				x^2 + \frac{1}{4} & x \neq \frac{1}{2},\\
				\frac{1}{2} & x = \frac{1}{2}
			\end{cases}
		\end{equation*}
		and then proceeded to \textbf{{\color{red}incorrectly}} differentiate it piecewise as
		\begin{equation*} 
			f'(x) = \begin{cases}
				2x & x \neq \frac{1}{2},\\
				0 & x = \frac{1}{2}.
			\end{cases}
		\end{equation*}
		The above is \textbf{absurd}.

		Firstly, note that the function defined above is actually just $f(x) = x^2 + \frac{1}{4}$ and as such, is infinitely differentiable everywhere.

		Secondly, the reason you cannot do that is because for you to differentiate ``piecewise,'' you need an interval around that point on which the function doesn't change definition. This is something I had mentioned in Tutorial 2.
		%
		\item In the same vein as above, you need to explicitly compute the first (and lack thereof of second) derivative at $\frac{1}{2}$ using the first principle (i.e., LHD and RHD). You cannot differentiate piecewise and argue using limits without additional justification.
		%
		\item Recall the slide in recap where I had put a ``please.'' I had requested you to remember that a function can be differentiable at a point without the derivative being continuous at that point. As such, all arguments using just limits of $f'$ (or $f''$) have been discarded. (You could possibly argue using that but that needs more justification such as Darboux's theorem or something else.)
		%
		\item Some of you had written something like \\
		\begin{blockquote}
		
		\begin{equation*} 
			f''(x) \vcentcolon= \begin{cases}
				4 & x \ge \frac{1}{2},\\
				2 & x < \frac{1}{2}
			\end{cases}
		\end{equation*}
		and hence, $f''\left(\frac{1}{2}\right)$ does not exist because the left and right hand limits of $f''$ don't agree. 
		\end{blockquote}

		This is \textbf{absurd}. To begin with, refer to the previous point.

		Even keeping that aside: You, technically, are already saying $f''\left(\frac{1}{2}\right) = 4$ in the above and hence, according to \emph{you}, $f'$ is indeed differentiable at $\frac{1}{2}.$
		%
		\item Some of you have tried to conclude convexity by saying that $f''(x) \ge 0.$ Firstly, $f''$ doesn't even exist at $\frac{1}{2}.$ Secondly, if you are arguing by saying that ``The second derivative is positive, wherever it exists,'' that is also not correct. For example, consider $g:\mathbb{R}\to\mathbb{R}$ defined as
		\begin{equation*} 
			g(x) \vcentcolon= \begin{cases}
				(x + 1)^2 - 1 & x < 0,\\
				x^2 & x \ge 0.
			\end{cases}
		\end{equation*}
		$g''$ is defined on $\mathbb{R}\setminus\{0\}$ and is positive there. However, $g$ is not convex.
		%
	\end{enumerate}
	%
	%
	\item This was fine for most of you. Do justify why $f$ has the properties using $f'$ and $f''$ (if you pick a twice differentiable function, like I did). Don't draw graphs.

\end{enumerate}
\end{document}